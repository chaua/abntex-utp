% =======================================================================
% =                                                                     =
% = ABNTEX - UTP                                                        =
% =                                                                     =
% =======================================================================
% -----------------------------------------------------------------------
% Author: Chaua Queirolo
% Data:   01/07/2017
% -----------------------------------------------------------------------
\documentclass[12pt,oneside,a4paper,chapter=TITLE,section=TITLE,sumario=tradicional]{abntex2}

% Regras da abnt
\usepackage{packages/abnt-UTP}
\usepackage{lipsum}

% =======================================================================
% =                                                                     =
% = DADOS DO TRABALHO                                                   =
% =                                                                     =
% =======================================================================

% Informações de dados para CAPA e FOLHA DE ROSTO
\titulo{Título do Trabalho}

\autor{Nome do Aluno}

\orientador{Prof. Nome do Professor}

\preambulo{Trabalho de Conclusão de Curso apresentado ao curso de Bacharelado 
em Ciência da Computação da Faculdade de Ciências Exatas e de Tecnologia da 
Universidade Tuiuti do Paraná, como requisito à obtenção ao grau de Bacharel.}

\instituicao{Universidade Tuiuti do Paraná}
\local{Curitiba}
\data{2017}

% =======================================================================
% =                                                                     =
% = DOCUMENTO                                                           =
% =                                                                     =
% =======================================================================
\begin{document}

% -----------------------------------------------------------------------
% -                                                                     -
% - ELEMENTOS PRÉ-TEXTUAIS                                              -
% -                                                                     -
% -----------------------------------------------------------------------

% Capa e folha de rosto
\imprimircapa
\imprimirfolhaderosto

% Resumo
\begin{resumo}
    Texto do resumo.
    
    \palavraschave{Palavra 1, Palavra 2, Palavra 3}    
\end{resumo}

% Listas
\listadefiguras
\listadegraficos
\listadetabelas
\listadequadros
\listadecodigos
\listadealgoritmos

% Lista de siglas
\begin{siglas}
  \item[ABNT] Associação Brasileira de Normas Técnicas
\end{siglas}
% ---

% Lista de símbolos
\begin{simbolos}
  \item[$ \Gamma $] Letra grega Gama
  \item[$ \Lambda $] Lambda
  \item[$ \zeta $] Letra grega minúscula zeta
  \item[$ \in $] Pertence
\end{simbolos}

% Sumario
\sumario

% -----------------------------------------------------------------------
% -                                                                     -
% - ELEMENTOS TEXTUAIS                                                  -
% -                                                                     -
% -----------------------------------------------------------------------
% Inicia a numeracao das páginas
\textual

% -----------------------------------------------------------------------
% -----------------------------------------------------------------------
\chapter{Introdução}
\label{cap:introducao}


A introdução tem a função de fazer a abertura do trabalho. Para tanto, deve 
apresentar a delimitação do assunto tratado, os objetivos da pesquisa e outros 
elementos necessários para situar o leitor. É possível que apresente:

\begin{lista}
    \item a problematização, a motivação e a justificativa da escolha do tema;
    \item o problema de pesquisa e suas hipóteses, se houver,
    \item a metodologia da pesquisa;
    \item o referencial teórico e, ainda,
    \item os tópicos principais do desenvolvimento, dando o roteiro ou
    ordem de exposição no decorrer da parte textual.
\end{lista}

É importante que a introdução seja um texto claro, conciso e interessante, pois 
é por meio dessa abertura que se consegue prender a atenção do leitor e motivá- 
lo à leitura do desenvolvimento do trabalho. Indica-se que o texto de 
introdução seja inteiro, isto é, sem divisões com subtítulos para cada um dos 
elementos que apresenta.

Modernamente, admite-se o uso da primeira pessoa do singular ou, no caso de uma 
equipe, a primeira pessoa do plural. Entretanto, é difícil redigir dessa forma 
sem incorrer no excesso de subjetivismo. De qualquer modo, o importante é que a 
redação seja sempre coerente: se começar com a primeira do singular não mude 
para uma forma impessoal no meio do texto. E vice-versa.

% -----------------------------------------------------------------------
% -----------------------------------------------------------------------
\chapter{Fundamentação Teórica}
\label{cap:fundamentacao-teorica}

Consiste na apresentação de teorias e seus teóricos aos quais a pesquisa se 
reportará. Isto é, explicita as linhas teóricas que serão seguidas como 
referência à pesquisa. O Referencial Teórico é um espaço no projeto de pesquisa 
que poderá apresentar diferentes agrupamentos de informações, podendo estar 
desmembrado em subpartes, tais como as teorias de base, uma revisão de 
literatura, alguns trabalhos relacionados, dentre outros.

% -----------------------------------------------------------------------
% -----------------------------------------------------------------------
\chapter{Revisão da Literatura}
\label{cap:revisao-literatura}

Expressão das principais conclusões e resultados que outros pesquisadores 
obtiveram sobre o tema escolhido, mostrando como se corroboram, se confirmam, e 
como divergem e discordam entre si. Esse levantamento do estado atual de 
conhecimento do tema – também chamado de ``estado da arte'' – é fundamental para 
que o esforço do trabalho de pesquisa não resulte apenas em conhecimentos que 
já haviam sido alcançados por outros pesquisadores, tecnologias já 
desenvolvidas e informações já disponibilizadas pela Ciência.

Trabalhos relacionados são aquelas pesquisas que foram desenvolvidas com maior 
proximidade ao que se pretende fazer, não apenas pela similaridade do tema e/ou 
dos modelos teóricos, mas, também pela metodologia escolhida. Deve apresentar 
os autores, um descritivo de suas pesquisas e como estas se relacionam com o 
trabalho que se pretende realizar.

% -----------------------------------------------------------------------
% -----------------------------------------------------------------------
\chapter{Metodologia}
\label{cap:metodologia}

Consiste em detalhar a operacionalização da pesquisa, informando como, onde e 
com quê o estudo será conduzido. O elemento ``Metodologia'' pode informar, por 
exemplo:

\begin{lista}
    \item  o tipo de pesquisa e os passos para a sua execução;

    \item a seleção da equipe técnica, se houver, e dos materiais
necessários;

    \item as intenções referentes à escolha dos métodos, das técnicas
e dos instrumentos de coleta de dados (entrevista, formulário,
observação, testes ou outros);

    \item os locais da pesquisa de campo;

    \item a seleção de amostras;

    \item a forma de dispor os dados, isto é, a tabulação;

    \item a forma e os passos para análise e interpretação das
informações obtidas;

    \item o tipo de documento que resultará da pesquisa (monografia,
artigo, relatório técnico, ou outro).
\end{lista}

% -----------------------------------------------------------------------
% -----------------------------------------------------------------------
\chapter{Resultados Exeperimentais}
\label{cap:resultados}

Validação da metodologia apresentada no \autoref{cap:metodologia}.

% -----------------------------------------------------------------------
% -----------------------------------------------------------------------
\chapter{Cronograma}
\label{cap:cronograma}

Especifica o tempo que será dispensado para a pesquisa, planejado em cada uma 
das etapas. Sua estrutura pode ser apresentada em meses ou semanas, conforme os 
requisitos da instituição (tempo mínimo exigido e máximo permitido) e as etapas 
variam de acordo com as diferentes áreas do saber.

% -----------------------------------------------------------------------
% -----------------------------------------------------------------------
\chapter{Elementos de apoio}
\label{cap:apoio}

% -----------------------------------------------------------------------
\section{Ilustrações}
\label{cap:ilustracoes}

Ilustrações são elementos cuja função é complementar ao texto: são explicativas 
e informativas, não podendo apenas adornar ou enfeitar o trabalho. Fazem parte 
das ilustrações: desenhos, esquemas, fluxogramas, fotografias, gráficos, mapas, 
organogramas, plantas, quadros, retratos, figuras, imagens e outros.

A ilustração deve ser anunciada no texto – chamada pelo seu nmero (algarismos 
arábicos) – e inserida o mais próximo possível do trecho a que se refere.
Qualquer que seja o tipo de ilustração, sua identificação aparece na parte 
superior, precedida da palavra designativa (Figura, Mapa, Gráfico, etc.), 
seguida de seu número de ordem de ocorrência no texto, em algarismos arábicos, 
hífen ou travessão e do respectivo título.

Quando a ilustração for elaborada pelo(s) autor(es) do trabalho, deverá 
aparecer ``o próprio autor'' ou, no caso de trabalho em equipe, ``os próprios 
autores''. A \autoref{fig:grafico} apresenta um exemplo de gráfico. A 
\autoref{fig:figura} apresenta um exemplo de figura centralizada, enquanto a 
\autoref{fig:subfigura} apresenta exemplos de subfiguras.

\begin{grafico}[htb]
    \legenda[fig:grafico]{Exemplo de gráfico}
    \fig{scale=0.6}{imagens/teste}
    \fonte{\citeonline[p. 24]{araujo2012}}
\end{grafico}

\begin{figure}[htb]
    \legenda[fig:figura]{Exemplo de figura}
    \fig{scale=0.6}{imagens/teste}
    \fonte{\citeonline[p. 24]{araujo2012}}
\end{figure}

\begin{figure}[htb]
    \legenda[fig:subfigura]{Exemplo de várias subfiguras}
    \sfig{scale=0.3}{imagens/teste}\hfil
    \sfig{scale=0.3}{imagens/teste}\hfil
    \sfig{scale=0.3}{imagens/teste}
    
    \lfig[s:a3]{scale=0.3}{imagens/teste}{zzzz}\hfil
    \lfig[s:a3]{scale=0.3}{imagens/teste}{zzzz}\hfil
    \lfig[s:a4]{scale=0.3}{imagens/teste}{yyyy}
    
    \fonte{teste}
\end{figure}

% -----------------------------------------------------------------------
\section{Tabelas e Quadros}
\label{sec:tabelas}

As tabelas não são consideradas ilustrações, mas, sim, elementos demonstrativos 
de síntese. Por serem autossuficientes, não complementam o texto, isto porque 
já fazem parte dele como uma organização estrutural esquematizada. Segundo o 
IBGE (1993), as tabelas apresentam dados e/ou informações oriundos de 
tratamento estatístico e sua inserção no decorrer dos trabalhos segue as mesmas 
regras aplicadas para as ilustrações (identificação na parte superior com 
número e título, e fonte de referência na parte inferior em letra menor).

As tabelas podem ser inseridas no texto ou em anexo (principalmente as de 
formato grande, que ocupam uma página inteira ou mais). Recomenda-se incluir a 
observação ``continua...'' e ``... continuação'' nas respectivas partes, quando 
a tabela ocupar mais de uma página. Quando inseridas no texto, devem ser 
alinhadas às margens laterais ou centralizadas, se apresentarem formato pequeno.

Em suas delimitações, são usados traços horizontais para destacar o cabeçalho, 
bem como traço horizontal final. Indica-se a delimitação, no alto e em baixo, 
por traços horizontais grossos, preferencialmente. Não deve ser delineada à 
direita e à esquerda, por traços verticais e é facultativo o emprego de traços 
verticais para separação das colunas no corpo da tabela.

\begin{table}[htb]
    \legenda[tab:exemplo]{Exemplo de tabela}
    \begin{tabular}{c|ccc}
        \hline\
        \textbf{Pessoa} & \textbf{Idade} & \textbf{Peso} & \textbf{Altura} \\ 
        \hline\hline
        Marcos & 26    & 68   & 178    \\ 
        Ivone  & 22    & 57   & 162    \\ 
        ...    & ...   & ...  & ...    \\ 
        Sueli  & 40    & 65   & 153    \\ \hline
    \end{tabular}
    
    \fonteautor
\end{table}
 
Quando houver informações ou dados numéricos que não necessitem de cálculos 
(por exemplo, características, propriedades, relações, etc.), poderão ser 
utilizados os quadros. Nestes, os traços contornam toda a tabela.

\begin{lista}
	\item novo
	\item asd
	\item asd
	\item asd
	\item asd
\end{lista}

\begin{quadro}[htb]
    \legenda[quadro:exemplo]{Exemplo de quadro}
    \begin{tabular}{|c||c|c|c|}
        \hline
        \textbf{Pessoa} & \textbf{Idade} & \textbf{Peso} & \textbf{Altura} \\ 
        \hline\hline
        Marcos & 26    & 68   & 178    \\ \hline
        Ivone  & 22    & 57   & 162    \\ \hline
        ...    & ...   & ...  & ...    \\ \hline
        Sueli  & 40    & 65   & 153    \\ \hline
    \end{tabular}
    
    \fonte{\cite{EIA649B}}
\end{quadro}

% -----------------------------------------------------------------------
\section{Equações}
\label{sec:equações}

Para facilitar a leitura, equações e fórmulas devem ser destacadas no texto e, 
se necessário, numeradas com algarismos arábicos entre parênteses, alinhados à 
direita. Na sequência normal do texto, é permitido o uso de uma entrelinha 
maior que comporte seus elementos (expoentes, índices, entre outros). A 
\autoref{eq:exemplo} apresenta um exemplo de equação.

\begin{equation}
\label{eq:exemplo}
C_{(A,B)} = \{ p \in
A\;|\;[(\overrightarrow{q_i-c}){\cdot}{\vec{n}}_c][(\overrightarrow{q_j-c}){\cdot}{\vec{n}}_c]
< 0 \}
\end{equation}

% -----------------------------------------------------------------------
% -----------------------------------------------------------------------
\section{Códigos e Algoritmos}
\label{sec:codigos}

Os códigos e algoritmos podem ser inseridos no texto usando comandos 
\texttt{codigo} e \texttt{algoritmo}, respectivamente. O \autoref{cod:fib} 
apresenta um exemplo de código em C.

\begin{codigo}[htb]
    \legenda[cod:fib]{Calcula Fibonacci}
    \begin{lstlisting}[language=C]
    int main() {
        int n, first = 0, second = 1, next, c;
        
        printf("Enter the number of terms\n");
        scanf("%d", &n);
        
        printf("First %d terms of Fibonacci series are :-\n", n);
        
        for (c = 0; c < n; c++){
            if (c <= 1) next = c;
            else {
                next = first + second;
                first = second;
                second = next;
            }
            printf("%d\n",next);
        }
        
        return 0;
    }
    \end{lstlisting}
    
    \fonteautor
\end{codigo}


% -----------------------------------------------------------------------
% -----------------------------------------------------------------------
\chapter{Conclusão}

É a parte final do trabalho, na qual se apresentam as conclusões 
correspondentes aos objetivos e às hipóteses: informa se os objetivos foram 
alcançados ou não – seguidos de justificativas e explicações caso os mesmos não 
tenham sido alcançados – bem como se as hipóteses foram negadas ou 
corroboradas. É possível que se apresentem também:

\begin{lista}
    \item comentários relativos aos resultados obtidos, fechando o raciocínio 
    por meio de um processo dedutivo,
    
    \item a importância dos resultados obtidos,
    
    \item a projeção da pesquisa, com estimativas para o uso dos resultados,
    
    \item a repercussão, informando quem será beneficiado e em quê,
    
    \item as limitações do trabalho, mostrando suas fragilidades ou
insuficiências,
    \item as dificuldades encontradas no decorrer da pesquisa, e
    \item indicações para trabalhos futuros, para a continuidade da
pesquisa pelo próprio autor e por outros.
\end{lista}

Veja \autoref{apendice:teste}.

% ----------------------------------------------------------
% ELEMENTOS PÓS-TEXTUAIS
% ----------------------------------------------------------
%\postextual
% ----------------------------------------------------------

% ----------------------------------------------------------
% Referências bibliográficas
% ----------------------------------------------------------
\bibliography{referencias}

% ----------------------------------------------------------
% Apêndices
% ----------------------------------------------------------
% Material complementar preparado pelo autor
\apendice[apendice:teste]{TESTE}

% ----------------------------------------------------------
% Anexos
% ----------------------------------------------------------
% Material complementar nao preparado pelo autor
\anexo[apendice:teste1]{TESTE}


\end{document}
