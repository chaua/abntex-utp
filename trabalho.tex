% =================================================================================================
% = ///////////////////////////////////////////////////////////////////////////////////////////// =
% = ///                                                                                       /// =
% = /// ABNTEX - UTP                                                                          /// =
% = ///                                                                                       /// =
% = ///////////////////////////////////////////////////////////////////////////////////////////// =
% =================================================================================================
% -------------------------------------------------------------------------------------------------
% - Author: Chauã Queirolo
% - Data:   01/10/2024
% -------------------------------------------------------------------------------------------------
\documentclass[
    12pt
    ,oneside
    ,a4paper
    ,chapter=TITLE
    ,section=TITLE
    ,sumario=abnt-6027-2012]{abntex2}

% Carrega o pacote da abnt com as adapatções para normas da Tuiuti
\usepackage{packages/abnt-UTP}

% =================================================================================================
% -------------------------------------------------------------------------------------------------
% -                                                                                               -
% - INFORMAÇÕES GERAIS                                                                            -
% -                                                                                               -
% -------------------------------------------------------------------------------------------------
% =================================================================================================

% Informações para CAPA e FOLHA DE ROSTO
\titulo{Título do Trabalho}

\autor{Nome Completo do Aluno}

\orientador{Prof. Nome do Professor}

\preambulo{Trabalho de Conclusão de Curso apresentado ao curso de Bacharelado em Ciência da
Computação da Faculdade de Ciências Exatas e de Tecnologia da Universidade Tuiuti do Paraná,
como requisito à obtenção ao grau de Bacharel.}

\instituicao{Universidade Tuiuti do Paraná}
\local{Curitiba}
\data{\the\year}

% =================================================================================================
% -------------------------------------------------------------------------------------------------
% -                                                                                               -
% - DOCUMENTO                                                                                     -
% -                                                                                               -
% -------------------------------------------------------------------------------------------------
% =================================================================================================
\begin{document}

% =================================================================================================
% -------------------------------------------------------------------------------------------------
% - ELEMENTOS PRÉ-TEXTUAIS                                                                        -
% -------------------------------------------------------------------------------------------------
% =================================================================================================

% -------------------------------------------------------------------------------------------------
% Cria a capa e a folha de rosto
\imprimircapa
\imprimirfolhaderosto

% Resumo do trabalho
\begin{resumo}
    O resumo deve ser informativo e conciso, apresentando os principais elementos do trabalho: o
    tema principal, os objetivos, os métodos, os resultados e as conclusões. Para trabalhos
    acadêmicos, ele deve ser escrito em um único parágrafo e possuir entre~150 a~500 palavras.
    A linguagem o do resumo deve utilizar frases curtas e impessoais, sem figuras de linguagem
    ou citações, preferencialmente na terceira pessoa do singular. Devem ser evitados adjetivos
    e advérbios que não agreguem informações importantes ao texto.
    Após o resumo, devem ser apresentadas de~3 a~5 palavras-chave, separadas por ponto e
    vírgula finalizada por ponto final.

    % O comando \palavraschave inclui automáticamente o ponto final
    \palavraschave{Palavra-chave 1; Palavra-chave 2; Palavra-chave 3}
\end{resumo}


% -------------------------------------------------------------------------------------------------
% Listas de ilustrações, tabelas, siglas, etc.
% - Inclua apenas as listas dos elementos que serão utilizados no trabalho.
% - A lista de símbolos e siglas devem ser editadas manualmente 
\listadefiguras
\listadegraficos
\listadetabelas
\listadequadros
\listadecodigos
\listadealgoritmos


% Lista de siglas
% - As siglas deve ser organizadas em ORDEM ALFABÉTICA
% - Inclua apenas as siglas que aparecem no trabalho
\begin{siglas}
    \item[ABNT]   Associação Brasileira de Normas Técnicas
    \item[IA]     Inteligência Artificial
    \item[DBMS]   \textit{Database Management System}
    \item[GPU]    \textit{Graphics Processing Unit}
    \item[HAL]    \textit{Heuristically programmed ALgorithmic computer}
    \item[POO]    Programação Orientada a Objetos
    \item[UI]     \textit{User Interface} (Interface do Usuário)
    \item[UX]     \textit{User Experience} (Experiência do Usuário)
\end{siglas}


% Lista de símbolos
\begin{simbolos}
    \item[$ \Sigma $]   Somatório
    \item[$ \prod $]    Produtório
    \item[$ \in $]      Pertence 
    \item[$ \notin $]   Não pertence
    \item[$ \mathcal{O}(n) $] Notação Big O 
    \item[$ \forall $]  Para todo (quantificador universal)
    \item[$ \exists $]  Existe (quantificador existencial)
\end{simbolos}

% Criação do sumário do trabalho
\sumario

% =================================================================================================
% -------------------------------------------------------------------------------------------------
% - ELEMENTOS TEXTUAIS                                                                            -
% -------------------------------------------------------------------------------------------------
% =================================================================================================

% Reinicia a numeração das páginas
\textual

% -------------------------------------------------------------------------------------------------
% -------------------------------------------------------------------------------------------------
% 1. INTRODUÇÃO
% -------------------------------------------------------------------------------------------------
% -------------------------------------------------------------------------------------------------
\chapter{Introdução}
\label{cap:introducao}

A introdução é uma seção fundamental em qualquer trabalho acadêmico, pois estabelece as bases para a pesquisa a ser realizada. Nela, o autor deve apresentar o tema que será investigado, situando-o no contexto atual e destacando sua relevância para o campo de estudo. Conforme destacado por \citeonline{severino}, a introdução deve fornecer uma visão geral clara e concisa, permitindo ao leitor compreender as motivações e os objetivos da pesquisa.

A primeira etapa na elaboração da introdução é a contextualização do tema. É necessário que o autor apresente uma descrição detalhada do cenário em que o problema de pesquisa está inserido, evidenciando a importância do assunto em questão. De acordo com \citeonline{lakatos}, essa contextualização é essencial para que o leitor perceba a pertinência da investigação, além de indicar as lacunas existentes na literatura que o trabalho busca preencher.

Em seguida, a introdução deve incluir a formulação do problema de pesquisa. Esta etapa é crucial, uma vez que a clareza e especificidade da pergunta norteadora influenciam diretamente o desenvolvimento do trabalho. Como mencionado por \citeonline{creswell}, a formulação adequada do problema estabelece os limites da pesquisa e direciona as investigações subsequentes. Portanto, é imperativo que o problema seja apresentado de maneira explícita e compreensível.

Outro aspecto relevante da introdução é a definição dos objetivos da pesquisa. Os objetivos devem ser claramente enunciados, subdividindo-se em um objetivo geral e objetivos específicos. O objetivo geral reflete a meta maior do estudo, enquanto os objetivos específicos detalham as etapas necessárias para alcançar esse propósito. Essa estrutura ajuda a guiar o leitor ao longo da pesquisa, fornecendo um panorama sobre o que se pretende investigar~\cite{gil}.

Adicionalmente, a introdução deve abordar de forma breve a metodologia que será utilizada ao longo do trabalho. Embora a descrição completa da metodologia seja mais adequada em uma seção específica, oferecer uma visão geral na introdução ajuda o leitor a compreender como o autor planeja abordar o problema de pesquisa. Segundo \citeonline{koche}, a metodologia é a estratégia que o pesquisador utiliza para alcançar os objetivos e responder ao problema formulado.

A introdução de um trabalho acadêmico deve ser redigida com clareza e objetividade, abordando os elementos centrais da pesquisa, como a contextualização do tema, a formulação do problema, a justificativa, os objetivos e a metodologia. Além disso, a introdução pode fazer uso de elementos não textuais, como ilustrações e tabelas, para facilitar a compreensão da metodologia que será desenvolvida. No entanto, a introdução deve ser linear e não subdividida, ou seja, a divisão em subseções sugere detalhamento, o que deve ser reservado para as seções específicas.

Finalmente, a introdução deve concluir com uma breve descrição da estrutura do trabalho, indicando como os capítulos estão organizados. Essa abordagem permite ao leitor ter uma compreensão clara do fluxo do texto e dos tópicos que serão discutidos ao longo da pesquisa. A seguir, apresenta-se a estrutura do trabalho.

Este trabalho está organizado como segue. 
O Capítulo~\ref{cap:organizacao} descreve a estrutura típica de trabalhos acadêmicos de graduação. 
O Capítulo~\ref{cap:citacao} aborda as normas para formatação e uso de citações, distinguindo entre citação direta, indireta e de fontes. 
O Capítulo~\ref{cap:ilustracoes} orienta sobre a inserção de ilustrações e gráficos, enquanto o Capítulo~\ref{cap:tabelas} esclarece a diferença entre tabelas e quadros, com exemplos práticos. 
O Capítulo~\ref{cap:codigos} trata da apresentação de códigos e algoritmos.
O Capítulo~\ref{cap:equacoes} explora o uso de equações, e o Capítulo~\ref{cap:alineas} discute as regras para o uso de alíneas. 
Por fim, o Capítulo~\ref{cap:cronograma} trata da formatação de cronogramas, e o Capítulo~\ref{cap:conclusao} apresenta as considerações finais.


% -------------------------------------------------------------------------------------------------
% -------------------------------------------------------------------------------------------------
% -------------------------------------------------------------------------------------------------



% -------------------------------------------------------------------------------------------------
% -------------------------------------------------------------------------------------------------
% 2. ORGANIZAÇÃO
% -------------------------------------------------------------------------------------------------
% -------------------------------------------------------------------------------------------------
\chapter{Organização do Trabalho}
\label{cap:organizacao}

Este capítulo apresenta as principais seções de um trabalho acadêmico, discute a hierarquia de seções  e fornece exemplos de como implementar a sua organização. A organização de um trabalho acadêmico é um ponto fundamental para garantir a clareza e a estrutura lógica do texto~\cite{severino}.

Normalmente, um trabalho acadêmico possui a seguinte divisão de capítulos:

\begin{alineas}
    \item Introdução: estabelece o contexto do trabalho e apresenta o problema de pesquisa;
    \item Fundamentação teórica: explora os conceitos e teorias que embasam a pesquisa, proporcionando uma base teórica para a compreensão do tema;
    \item Revisão da Literatura: analisa e discute estudos anteriores relacionados ao tema, identificando lacunas que a pesquisa busca preencher;
    \item Metodologia: descreve os métodos e técnicas utilizados para conduzir a pesquisa, permitindo a reprodução do estudo por outros pesquisadores;
    \item Resultados Experimentais: apresenta e analisa os dados obtidos durante a pesquisa, evidenciando os achados e sua relevância para o tema investigado;
    \item Conclusão: resume os principais resultados, discute suas implicações, e sugere direções para futuras pesquisas ou aplicações práticas.
\end{alineas}

A organização adequada de um trabalho acadêmico não apenas melhora a legibilidade, mas também facilita a navegação pelo documento. O uso correto dos comandos de seção, subseção e subsubseção ajuda a estruturar o conteúdo de forma lógica e coesa, orientando o leitor através dos diferentes tópicos abordados. Além disso, a numeração automática das seções e subseções torna o trabalho mais profissional e alinhado às normas acadêmicas~\cite{gil}.



% -------------------------------------------------------------------------------------------------
% -------------------------------------------------------------------------------------------------
% -------------------------------------------------------------------------------------------------


% -------------------------------------------------------------------------------------------------
% -------------------------------------------------------------------------------------------------
% 3. CITAÇÃO
% -------------------------------------------------------------------------------------------------
% -------------------------------------------------------------------------------------------------
\chapter{Citação}
\label{cap:citacao}

Este capítulo apresenta os principais tipos de citação empregados em trabalhos acadêmicos, detalhando suas características e a importância de sua correta aplicação para fundamentar teoricamente a pesquisa. Citações diretas e indiretas são essenciais para a construção de uma argumentação sólida, e o uso de ferramentas como o BibTeX facilita a organização e padronização das referências bibliográficas. Ao final, são discutidas as melhores práticas para citação de livros, artigos, capítulos de livros, teses, dissertações e fontes online, com vistas à preservação da integridade acadêmica.

\section{Tipos de Citações}

As citações são classificadas em dois tipos principais: (1) diretas e (2) indiretas, sendo ambas complementares no desenvolvimento da argumentação e fundamentação teórica de um trabalho.

\subsection{Citação Direta}

A citação direta consiste na reprodução literal das palavras de um autor. Segundo Gil~\cite{gil}, ela é utilizada quando se deseja destacar uma formulação precisa ou quando o uso exato das palavras é fundamental para a compreensão do argumento. No entanto, seu uso deve ser comedido e sempre referenciado adequadamente.

As citações diretas curtas podem ser inseridas com aspas e acompanhadas da devida referência, como demonstrado: ``O conhecimento científico é sistemático e controlado''~\cite{gil}. Quando a citação excede três linhas, deve ser formatada em bloco, sem aspas, como no exemplo a seguir:

\begin{citacao}
    O conhecimento científico é um conhecimento crítico, que tenta resolver os problemas que surgem em sua busca pela compreensão dos fenômenos, submetendo suas hipóteses e teorias a rigorosos testes de falseabilidade~\cite{gil}.
\end{citacao}

Por se tratar de uma reprodução exata, a citação direta deve preservar a ortografia, a pontuação e o uso de maiúsculas do texto original. Qualquer alteração ou omissão no texto original deve ser indicada com colchetes. Mesmo os erros presentes no texto original não devem ser corrigidos~\cite{utp_normas2024}. Para isso, algumas convenções são seguidas, como:

\begin{alineas}
    \item {[...]}: indica a omissão de parte do texto original;
    \item {[sic]}: sinaliza um erro ou incoerência no texto original, significando "assim mesmo";
    \item {[grifo meu]}, [grifo nosso], [sem grifo no original]: utilizados quando o autor que cita adiciona ênfase (negrito, sublinhado, etc.) que não estava no texto original;
    \item {[grifo do autor]}: indica que o grifo já fazia parte do texto original citado;
    \item Alterações ou comentários feitos na citação direta devem ser incluídos entre colchetes: [ ].
\end{alineas}

Citações em idiomas estrangeiros devem ser traduzidas, acompanhadas da devida referência. Quando a tradução for feita pelo próprio autor do trabalho, deve-se incluir a nota ``tradução do autor(a)'' e a versão original pode ser inserida como nota de rodapé. No entanto, em textos de caráter literário, é comum manter o idioma original, especialmente quando a tradução poderia comprometer o estilo ou a análise literária~\cite{leite, utp_normas2024}.

\subsection{Citação Indireta}

A citação indireta ocorre quando as ideias de outro autor são parafraseadas ou resumidas. Essa prática permite ao autor integrar diferentes perspectivas em seu próprio raciocínio. Segundo \citeonline{creswell}, a citação indireta é uma ferramenta importante para articular múltiplas fontes e construir uma análise mais crítica. As citações indiretas podem ser feitas de duas maneiras: (1)~explícitas ou (2)~implícitas. 

No caso da citação explícita, o nome do autor é mencionado no texto usando o comando \texttt{\textbackslash citeonline}, como no exemplo a seguir. De acordo com~\citeonline{creswell}, a pesquisa qualitativa depende de um processo interpretativo contínuo.

Já na citação indireta implícita, as ideias de um ou mais autores são sintetizadas e as citações são incluídas ao final da frase ou parágrafo usando o comando \texttt{\textbackslash cite}. as referências são inseridas ao final do parágrafo ou frase, sem menção direta ao autor no texto, como a seguir. A pesquisa qualitativa envolve um processo de reflexão constante e interpretação dos dados~\cite{creswell}.

\subsection{Citação de Fonte}

Ao utilizar figuras ou tabelas de outras fontes, ou até mesmo do próprio autor, é fundamental dar o devido crédito à fonte original. Isso garante o reconhecimento adequado do trabalho original e evita problemas de plágio. A citação da fonte de figuras e tabelas pode ser feita de maneira semelhante à citação de textos, usando o comando \texttt{\textbackslash fonte}. A Figura~\ref{fig:fontes} ilustra como as fontes devem ser citadas~\cite{utp_normas2024}.

\begin{figure}[htb]
    \legenda[fig:fontes]{Exemplo de citação de fontes}
    \fig{width=0.7\textwidth}{img/fontes}
    \fonte{\citeauthoronline{utp_normas2024}, \citeyear{utp_normas2024}, p. 23}
\end{figure}

O Código~\ref{cod:fontes} exemplifica como que os diferentes tipos de fontes devem ser incluídos no código para adequação às normas técnicas.

\begin{codigo}[htb]
\legenda[cod:fontes]{Diferentes tipos de citação em fontes}
\begin{lstlisting}
% Citação de autor
\fonte{\citeauthoronline{autor}, \citeyear{autor}, p. 99}

% Citação de autor-entidade
\fonte{\citeauthoronline{entidade}, \citeyear{entidade}, p. 105}

% Citação de site
\fonte{\citeauthoronline{site}, \citeyear{site}, Disponível em: \url{...}}

% Citação de autoria própria 
\fonteautor     % ou \fonteautora
\fonteautores   % ou \fonteautoras
\fonteautoria
\end{lstlisting}
\fonteautor
\end{codigo}

\section{Diferentes Tipos de Referências}

Além de conhecer os tipos de citação, é importante dominar os diversos formatos de referências utilizados em diferentes tipos de publicações. O uso do BibTeX permite automatizar a formatação dessas referências, garantindo padronização. A seguir, são apresentados os formatos para as principais fontes acadêmicas.

\subsection{Livros}

Referências de livros são amplamente utilizadas na fundamentação teórica, pois oferecem uma visão geral e detalhada sobre os conceitos e teorias fundamentais de uma área de estudo. Elas são essenciais para embasar o desenvolvimento dos argumentos do autor. A referência de livro pode ser inserida no arquivo BibTeX conforme apresentado no Código~\ref{cod:bibtex-book}.

\begin{codigo}[htb]
\legenda[cod:bibtex-book]{Estrutura BibTeX para citação de um livro}
\begin{lstlisting}
@book{autor,
    author    = {Sobrenome, Nome},
    title     = {Título do Livro},
    edition   = {Edição},
    publisher = {Editora},
    year      = {Ano},
    pages     = {Número de Paginas}
    address   = {Local de Publicacao}
}
\end{lstlisting}
\fonteautor
\end{codigo}

\subsection{Artigos de Periódicos}

Artigos científicos publicados em periódicos são uma fonte de informação atualizada e focada em tópicos específicos. Em um trabalho acadêmico, esses artigos são essenciais para manter a pesquisa em sintonia com as discussões mais recentes na área, e, por isso, sua correta citação é de extrema importância. A entrada BibTeX para artigos de periódicos é descrita no Código~\ref{cod:bibtex-article}.

\begin{codigo}[htb]
\legenda[cod:bibtex-article]{Estrutura BibTeX para citação de um artigo de periódico}
\begin{lstlisting}
@article{autor,
    author  = {Sobrenome, Nome},
    title   = {Título do Artigo},
    journal = {Nome do Periódico},
    volume  = {Volume},
    number  = {Número},
    pages   = {Páginas},
    year    = {Ano}
}
\end{lstlisting}
\fonteautor
\end{codigo}

\subsection{Capítulos de Livros}

Muitas vezes, é necessário citar apenas um capítulo de um livro organizado por diversos autores. Isso ocorre quando o capítulo em questão é escrito por um autor diferente do editor do livro, ou quando se deseja dar destaque a uma seção específica da obra. Isto pode ser feito utilizando a estrutura apresentada no Código~\ref{cod:bibtex-chapter}.

\begin{codigo}[htb]
\legenda[cod:bibtex-chapter]{Estrutura BibTeX para citação de um capítulo de livro}
\begin{lstlisting}
@incollection{autor,
    author    = {Sobrenome, Nome},
    title     = {Título do Capítulo},
    booktitle = {Título do Livro},
    publisher = {Editora},
    year      = {Ano},
    address   = {Local de Publicação},
    pages     = {Páginas do Capítulo}
}
\end{lstlisting}
\fonteautor
\end{codigo}

\subsection{Tese de Doutorado}

Teses de doutorado são referências relevantes em muitas áreas, especialmente porque contêm contribuições originais de pesquisa. Elas são uma fonte importante de conhecimento atualizado e inédito. A estrutura de citação para uma tese de doutorado em BibTeX pode ser vista no Código~\ref{cod:bibtex-phdthesis}.

\begin{codigo}[htb]
\legenda[cod:bibtex-phdthesis]{Estrutura BibTeX para citação de uma tese de doutorado}
\begin{lstlisting}
@phdthesis{autor,
    author  = {Sobrenome, Nome},
    title   = {Título da Tese},
    school  = {Nome da Universidade},
    year    = {Ano},
    address = {Local da Universidade}
}
\end{lstlisting}
\fonteautor
\end{codigo}

\subsection{Dissertação de Mestrado}

As dissertações de mestrado, assim como as teses de doutorado, são fundamentais para a pesquisa científica, por apresentarem investigações originais ou revisões aprofundadas de temas específicos. A estrutura para citar uma dissertação de mestrado é semelhante à de teses, como mostrado no Código~\ref{cod:bibtex-mastersthesis}.

\begin{codigo}[htb]
\legenda[cod:bibtex-mastersthesis]{Estrutura BibTeX para citação de uma dissertação de mestrado}
\begin{lstlisting}
@mastersthesis{autor,
    author  = {Sobrenome, Nome},
    title   = {Título da Dissertaçãoo},
    school  = {Nome da Universidade},
    year    = {Ano},
    address = {Local da Universidade},
    type    = {Dissertação de Mestrado}
}
\end{lstlisting}
\fonteautor
\end{codigo}

\subsection{Monografia de Graduação}

Monografias de graduação são trabalhos acadêmicos exigidos para a conclusão de cursos de graduação e podem ser relevantes como base de estudos introdutórios ou revisões bibliográficas. Para citá-las, o formato é apresentado no Código~\ref{cod:bibtex-monography} e é similar à de dissertações.

\begin{codigo}[htb]
\legenda[cod:bibtex-monography]{Estrutura BibTeX para citação de uma monografia de graduação}
\begin{lstlisting}
@mastersthesis{autor,
    author  = {Sobrenome, Nome},
    title   = {Título da Monografia},
    school  = {Nome da Universidade},
    year    = {Ano},
    address = {Local da Universidade},
    type    = {Monografia de Graduação}
}
\end{lstlisting}
\fonteautor
\end{codigo}

\subsection{Conferências}

Artigos apresentados em conferências acadêmicas também podem ser relevantes para a fundamentação teórica, principalmente em áreas de pesquisa emergentes. Esses artigos são revisados por pares e muitas vezes discutem temas inovadores. O Código~\ref{cod:bibtex-inproceedings} apresenta o formato de citação BibTeX de um artigo de conferência.

\begin{codigo}[htb]
\legenda[cod:bibtex-inproceedings]{Estrutura BibTeX para citação de um artigo de conferência}
\begin{lstlisting}
@inproceedings{autor,
    author    = {Sobrenome, Nome},
    title     = {Título do Artigo},
    booktitle = {Anais da Conferência},
    year      = {Ano},
    address   = {Local da Conferência},
    pages     = {Páginas},
    volume    = {Volume},
    number    = {Número}
}
\end{lstlisting}
\fonteautor
\end{codigo}

\subsection{Sites}

Sites também podem ser utilizados como fontes em pesquisas acadêmicas, desde que sejam confiáveis e possuam relevância no contexto da pesquisa. O uso de fontes \textit{online} é comum em pesquisas que abordam temas emergentes ou questões práticas. A citação de sites deve incluir a data de acesso, como mostrado no Código~\ref{cod:bibtex-online}, uma vez que conteúdos online podem ser modificados.

\begin{codigo}[htb]
\legenda[cod:bibtex-online]{Estrutura BibTeX para citação de um site}
\begin{lstlisting}
@online{pagina,
    author        = {Autor},
    organization  = {Organização},
    title         = {Título},
    year          = {Ano},
    url           = {URL da página},
    urlaccessdate = {05 out. 2024}
}
\end{lstlisting}
\fonteautor
\end{codigo}

A citação de verbetes da Wikipédia devem ser realizados conforme apresentado no Código~{cod:bibtex-wiki}. A citação será referenciada como apresentado em~\citeonline{wikipedia2024}.

\begin{codigo}[htb]
\legenda[cod:bibtex-wiki]{Estrutura BibTeX para citação de um artigo na Wikipédia}
\begin{lstlisting}
@inbook{wikipedia,
    title          = {Machine Learning},
    publisher      = {Wikimedia},
    booktitle      = {Wikip\'edia: a enciclop\'edia livre},
    year           = {2024},
    url            = {https://en.wikipedia.org/wiki/Machine_learning},
    urlaccessdate  = {05 out. 2024}
}
\end{lstlisting}
\fonteautor
\end{codigo}

\section{Considerações Finais}

A fundamentação teórica é a espinha dorsal de qualquer trabalho acadêmico, uma vez que é por meio dela que o autor constrói o suporte necessário para desenvolver suas ideias e hipóteses. O uso correto das citações e a inclusão de fontes confiáveis e pertinentes são fundamentais para assegurar a qualidade e a credibilidade da pesquisa.




% -------------------------------------------------------------------------------------------------
% -------------------------------------------------------------------------------------------------
% -------------------------------------------------------------------------------------------------



% -------------------------------------------------------------------------------------------------
% -------------------------------------------------------------------------------------------------
% 4. ILUSTRAÇÕES
% -------------------------------------------------------------------------------------------------
% -------------------------------------------------------------------------------------------------
\chapter{Ilustrações e Gráficos}
\label{cap:ilustracoes}

Este capítulo aborda o uso adequado de figuras e subfiguras em trabalhos acadêmicos, além de orientações sobre a inserção e formatação de gráficos. O uso de elementos visuais, como figuras e gráficos, é essencial para a clareza da apresentação de dados e conceitos. Em pesquisas que demandam a comunicação visual de resultados, esses elementos complementam a argumentação textual e auxiliam na compreensão do conteúdo~\cite{tufte1990envisioning}.

\section{Figuras}
\label{sec:figuras}

A inserção de figuras é feita utilizando o ambiente \texttt{figure}, que, em conjunto com o comando \texttt{\textbackslash fig}, permite a incorporação de imagens no documento. Cada figura deve ser acompanhada de uma legenda explicativa e, preferencialmente, ser referenciada no corpo do texto para que o leitor possa localizá-la facilmente~\cite{knuth1984texbook}. O Código~\ref{cod:figura} apresenta o exemplo de inserção de uma figura.

\begin{codigo}[htb]
\legenda[cod:figura]{Código para inclusão de uma figura no texto}
\begin{lstlisting}
\begin{figure}[htb]
    \legenda[fig:figura]{Legenda da figura}
    \fig{width=0.7\textwidth}{imagens/exemplo.png}
    \fonteautor
\end{figure}
\end{lstlisting}
\fonteautor
\end{codigo}

A Figura~\ref{fig:exemplo-fig} mostra um exemplo de figura usando os comandos do  Código~\ref{cod:figura}. 

\begin{figure}[htb]
    \legenda[fig:exemplo-fig]{Exemplo de figura}
    \fig{width=0.65\textwidth}{img/figura}
    \fonte{\citeauthoronline{xkcd-conclusion}, \citeyear{xkcd-conclusion}, Disponível em: \url{https://xkcd.com/1403/}}
\end{figure}


No Código~\ref{cod:figura}, a imagem é centralizada e redimensionada para 70\% da largura do texto. A legenda da figura é definida pelo comando \texttt{\textbackslash legenda}, e o rótulo é definido pelo parâmetro opcional (definido pelos colchetes). O rótulo permite que a figura seja referenciada no texto, como em ``veja a Figura~\ref{fig:exemplo-fig}''. 

O parâmetro ``[htb]'' indica as preferências de posicionamento da figura: ``h'' para o local exato do código, ``t'' para o topo da página, e ``b'' para a parte inferior~\cite{goossens1997latex}. 

\section{Subfiguras}
\label{sec:subfiguras}

As subfiguras são recomendadas em situações onde é necessário inserir múltiplas imagens agrupadas. Elas permitem organizar várias imagens em uma única figura, cada uma com sua legenda individual. Os comandos \texttt{\textbackslash sfig} e \texttt{\textbackslash lfig} podem ser utilizados para facilitar esse processo. O exemplo de código para inserção de subfiguras é apresentado no Código~\ref{cod:subfiguras}.

\begin{codigo}[htb]
\legenda[cod:subfiguras]{Código para inclusão de três subfiguras no texto}
\begin{lstlisting}
\begin{figure}[htb]
    \legenda[fig:subfigura]{Exemplo de varias subfiguras}
    % Linha 1: Figuras simples com label
    \sfig[sfig:fig1]{scale=0.3}{imagens/fig1}\hfil
    \sfig[sfig:fig2]{scale=0.3}{imagens/fig2}
    % Linha em branco move as próximas subfiguras para a linha seguinte

    % Linha 2: Figuras simples sem label
    \sfig{scale=0.3}{imagens/fig1}\hfil
    \sfig{scale=0.3}{imagens/fig2}

    % Linha 3: Figuras com legenda e com label
    \lfig[lfig:fig1]{scale=0.3}{imagens/fig1}{legenda 1}\hfil
    \lfig[lfig:fig2]{scale=0.3}{imagens/fig2}{legenda 2}

    % Linha 3: Figuras com legenda e sem label
    \lfig{scale=0.3}{imagens/fig1}{legenda 1}\hfil
    \lfig{scale=0.3}{imagens/fig2}{legenda 2}
    \fonteautor
\end{figure}
\end{lstlisting}
\fonteautor
\end{codigo}

No Código~\ref{cod:subfiguras}, três subfiguras são organizadas em uma linha, com espaçamento entre elas proporcionado pelo comando \textit{\textbackslash hfill}. As subfiguras podem ser referenciadas individualmente, como ``veja a Subfigura~\ref{sfig:fig1}'', ou em conjunto, como na ``veja a Figura~\ref{fig:subfigura1}''. Esse recurso é útil quando se deseja comparar diferentes imagens de maneira simultânea~\cite{goossens1997latex}.

A Figura~\ref{fig:subfigura1} apresenta um exemplo composição de subfiguras usando o comando \texttt{\textbackslash sfig}. Neste exemplo, todas as subfiguras foram organizadas em apenas uma linha.

\begin{figure}[htb]
    \legenda[fig:subfigura1]{Exemplo de subfiguras simples}
    \sfig[sfig:fig1]{scale=0.2}{img/fig1}\hfil
    \sfig{scale=0.2}{img/fig2}\hfil
    \sfig{scale=0.2}{img/fig3}\hfil
    \sfig{scale=0.2}{img/fig4}
    \fonte{\citeauthoronline{nintendo}, \citeyear{nintendo}, Disponível em: \url{https://www.nintendo.com}}
\end{figure}

O comando \texttt{\textbackslash lfig} deve ser utilizado quando for necessária a inclusão de legendas explicativas para cada uma das subfiguras. Este comando recebe um terceiro parâmetro que é a legenda referente a subfigura. A Figura~\ref{fig:subfigura2} mostra um exemplo usando este comando. Ambos os comandos podem ser combinados em uma mesma composição dentro do ambiente \texttt{figure}. 

\begin{figure}[htb]
    \legenda[fig:subfigura2]{Exemplo de subfiguras com legendas individuais}
    \lfig[sfig:mario]{scale=0.35}{img/fig1}{Mário}\hfil
    \lfig[sfig:luigi]{scale=0.35}{img/fig2}{Mário Verde}\hfil
    \lfig[sfig:peach]{scale=0.35}{img/fig3}{Princesa Peach}

    \lfig{scale=0.35}{img/fig4}{Cogumelo Toad}\hfil
    \lfig{scale=0.35}{img/fig5}{Rei Koopa}\hfil
    \lfig{scale=0.35}{img/fig6}{Yoshi}
    \fonte{\citeauthoronline{nintendo}, \citeyear{nintendo}, Disponível em: \url{https://www.nintendo.com}}
\end{figure}

\section{Gráficos}

A inserção de gráficos segue os mesmos princípios da inserção de figuras. O ambiente \texttt{grafico} é utilizado para acomodar gráficos gerados a partir de softwares de plotagem de dados ou imagens externas~\cite{mittelbach2004latex}. O Código~\ref{cod:grafico} apresenta um exemplo de inserção de gráfico e o Gráfico~\ref{fig:grafico} mostra o resultado do código.

\begin{codigo}[htb]
\legenda[cod:grafico]{Código para inclusão de um gráfico no texto}
\begin{lstlisting}
\begin{grafico}[htb]
    \legenda[fig:grafico]{Legenda do grafico}
    \fig{width=0.7\textwidth}{imagens/exemplo.png}
    \fonteautor
\end{grafico}
\end{lstlisting}
\fonteautor
\end{codigo}

\begin{grafico}[htb]
    \legenda[fig:grafico]{Exemplo de gráfico}
    \fig{width=0.7\textwidth}{img/grafico}
    \fonte{\citeauthor{xkcd-graph}, \citeyear{xkcd-graph}, Disponível em: \url{https://xkcd.com/1945/}}
\end{grafico}

Gráficos são recursos visuais valiosos para a representação de dados quantitativos e a comunicação clara de tendências ou padrões observados durante a pesquisa. Assim como nas figuras, é importante que os gráficos sejam acompanhados de legendas claras e referenciados no corpo do texto~\cite{tufte1990envisioning}.

\section{Considerações Finais}

O uso adequado de figuras, subfiguras e gráficos em trabalhos acadêmicos é fundamental para uma comunicação visual eficiente. Esses elementos não apenas ilustram os conceitos discutidos, mas também ajudam a sintetizar e evidenciar os resultados da pesquisa. Quando bem utilizados, eles aumentam a clareza e a coesão do trabalho, facilitando a compreensão pelo leitor~\cite{tufte1990envisioning, knuth1984texbook}.



% -------------------------------------------------------------------------------------------------
% -------------------------------------------------------------------------------------------------
% -------------------------------------------------------------------------------------------------



% -------------------------------------------------------------------------------------------------
% -------------------------------------------------------------------------------------------------
% 5. TABELAS E QUADROS
% -------------------------------------------------------------------------------------------------
% -------------------------------------------------------------------------------------------------
\chapter{Tabelas e Quadros}
\label{cap:tabelas}

Este capítulo aborda as técnicas de criação de tabelas e quadros, com a distinção de seus usos e estruturas. A apresentação organizada de dados é fundamental em trabalhos acadêmicos, e tanto as tabelas quanto os quadros desempenham um papel essencial na comunicação clara e precisa das informações. Contudo, eles diferem na natureza e no tipo de dado que apresentam.

\section{Diferença entre Tabelas e Quadros}

Embora os termos ``tabela'' e ``quadro'' sejam muitas vezes usados de forma intercambiável, há distinções claras entre esses elementos no contexto acadêmico. Tabelas são utilizadas principalmente para exibir dados quantitativos de forma estruturada, facilitando comparações entre diferentes valores ou categorias. Por outro lado, quadros organizam informações qualitativas, como definições ou classificações de conceitos, sendo mais descritivos e menos focados em dados numéricos~\cite{abnt2018}.

Assim, enquanto as tabelas proporcionam uma apresentação visual de dados comparativos ou numéricos, os quadros são apropriados para descrever ou categorizar informações qualitativas, o que evidencia a diferença fundamental entre ambos~\cite{chagas2016}.

\section{Tabelas}

As tabelas são criadas utilizando-se os ambientes \texttt{table} e \texttt{tabular}. O primeiro define o posicionamento e a legenda, enquanto o segundo organiza o conteúdo em linhas e colunas. O Código~\ref{cod:tabela} apresenta uma tabela com quatro colunas, com diferentes tipos de alinhamento: fixo com~3cm~(p), à esquerda~(l), centralizado~(c) e à direita~(d). A Tabela~\ref{tab:tabela} apresenta o resultado da tabela descrita no código.

\begin{codigo}[htb]
\legenda[cod:tabela]{Código para criação de uma tabela}
\begin{lstlisting}
\begin{table}[htb]
    \centering
    \legenda[tab:tabela]{Exemplo de uma tabela}
    \begin{tabular}{p{3cm}lcr}
        \toprule
        \textbf{Nome} & \textbf{Idade} & \textbf{Prova} & \textbf{Trabalhos} \\
        \midrule
        \midrule
        Joao    & 24 & 8,5 & 4,8 \\
        Maria   & 20 & 9,0 & 7,1 \\
        Pedro   & 22 & 4,8 & 9,9 \\
        \bottomrule
    \end{tabular}
    \fonteautor
\end{table}
\end{lstlisting}
\fonteautor
\end{codigo}

\begin{table}[htb]
    \centering
    \legenda[tab:tabela]{Exemplo de uma tabela}
    \begin{tabular}{p{3cm}lcr}
        \toprule
        \textbf{Nome} & \textbf{Idade} & \textbf{Prova} & \textbf{Trabalhos} \\
        \midrule
        \midrule
        Joao    & 24 & 8,5 & 4,8 \\
        Maria   & 20 & 9,0 & 7,1 \\
        Pedro   & 22 & 4,8 & 9,9 \\
        \bottomrule
    \end{tabular}
    \fonteautor
\end{table}



\section{Quadros}

Quadros, por sua vez, são mais adequados para a organização de informações qualitativas, sendo amplamente utilizados para categorizações ou descrições. Embora utilizem a mesma estrutura básica de uma tabela, sua função é distinta. O Código~\ref{cod:quadro} mostra um exemplo de quadro no qual são categorizados tipos de pesquisa, e o Quadro~\ref{tab:quadro} mostra o resultado deste código.

\begin{codigo}[htb]
\legenda[cod:quadro]{Código para criação de um quadro}
\begin{lstlisting}
\begin{quadro}[htb]
    \centering
    \legenda[tab:quadro]{Exemplo de Quadro com tipos de pesquisa}
    \begin{tabular}{|l|p{9cm}|}
        \hline
        \textbf{Tipo de Pesquisa} & \textbf{Descrição} \\ 
        \hline\hline
        Pesquisa Exploratória & Investigação inicial sobre um fenômeno, sem hipóteses definidas \\ \hline
        Pesquisa Descritiva   & Caracterização detalhada de um fenômeno, com hipóteses exploradas \\ \hline
        Pesquisa Explicativa  & Busca identificar as causas de um fenômeno \\         \hline
    \end{tabular}
    % Fix para ajustar a distancia da fonte (apenas para quadros)
    \vspace{0.2cm} 
    \fonteautor
\end{quadro}
\end{lstlisting}
\fonteautor
\end{codigo}


\begin{quadro}[htb]
    \centering
    \legenda[tab:quadro]{Exemplo de Quadro com tipos de pesquisa}
    \begin{tabular}{|l|p{9cm}|}
        \hline
        \textbf{Tipo de Pesquisa} & \textbf{Descrição} \\ 
        \hline\hline
        Pesquisa Exploratória & Investigação inicial sobre um fenômeno, sem hipóteses definidas \\ \hline
        Pesquisa Descritiva   & Caracterização detalhada de um fenômeno, com hipóteses exploradas \\ \hline
        Pesquisa Explicativa  & Busca identificar as causas de um fenômeno \\ \hline
    \end{tabular}
    \vspace{0.2cm} % Fix para ajustar a distância da fonte
    \fonteautor
\end{quadro}

O posicionamento de tabelas e quadros é controlado por meio de argumentos opcionais nos ambientes \texttt{table} e \texttt{quadro}. As opções mais comuns incluem:

\begin{alineas}
    \item \texttt{h} (\textit{here}): posiciona o objeto o mais próximo possível do ponto onde foi inserido no código;
    \item \texttt{t} (\textit{top}): coloca o objeto no topo da página;
    \item \texttt{b} (\textit{bottom}): coloca o objeto na parte inferior da página;
    \item \texttt{p} (\textit{page of floats}): insere o objeto em uma página dedicada apenas a elementos flutuantes, como tabelas e figuras.
\end{alineas}

Por exemplo, o argumento \texttt{[htb]}, utilizado nos Códigos~\ref{cod:tabela}~e~\ref{cod:quadro}, permite posicionar a tabela ``aqui''~(\textit{here}), mas com a flexibilidade de movê-la para o topo~(\textit{top}) ou parte inferior~(\textit{bottom}) da página, se necessário.

\section{Considerações Finais}

A correta utilização de tabelas e quadros é essencial para a apresentação organizada de dados e informações em trabalhos acadêmicos. Tabelas são mais apropriadas para dados quantitativos, enquanto quadros são utilizados para informações qualitativas e descritivas. A aplicação adequada desses elementos contribui significativamente para a clareza e profissionalismo do trabalho, facilitando a leitura e interpretação dos dados apresentados~\cite{knuth1984texbook}.


% -------------------------------------------------------------------------------------------------
% -------------------------------------------------------------------------------------------------
% -------------------------------------------------------------------------------------------------



% -------------------------------------------------------------------------------------------------
% -------------------------------------------------------------------------------------------------
% 6. CÓDIGOS
% -------------------------------------------------------------------------------------------------
% -------------------------------------------------------------------------------------------------
\chapter{Códigos e Algoritmos}
\label{cap:codigos}

Este capítulo apresenta o uso de algoritmos e trechos de código, abordando desde a inclusão básica até a personalização de estilos e numeração. A formatação clara e precisa de algoritmos e códigos é fundamental em áreas como ciência da computação, matemática e engenharias, onde a transparência dos métodos e a reprodutibilidade são essenciais para a validade dos resultados apresentados~\cite{knuth1984texbook}.


\section{Inclusão de Códigos}

Para incluir códigos de programação, o pacote mais comum é o \texttt{listings}. Esse pacote suporta várias linguagens de programação e permite a personalização do estilo de exibição, incluindo o realce de sintaxe, a numeração de linhas e a adição de comentários. O ambiente \texttt{codigo} formata a posição do código e localização da legenda e da fonte.

O Código~\ref{cod:codigo} apresenta um exemplo de código para inclusão de códigos-fonte e o Código~\ref{cod:fatorial} mostra o código renderizado.
Este código implementa uma função recursiva em Python para calcular o fatorial de um número. A sintaxe da linguagem Python é destacada automaticamente e o código é numerado.

\begin{codigo}[htb]
\legenda[cod:codigo]{Código para inclusão de códigos-fonte}
\begin{lstlisting}
\begin{codigo}[htb]
    \legenda[cod:codigo]{Código para inclusão de códigos-fonte}
    \begin{lstlisting}[language=Python]
    def fatorial(n):
        if n == 0:
            return 1
        else:
            return n * fatorial(n - 1)
    \end{lstlisting }
    \fonteautor
\end{codigo}
\end{lstlisting}
\fonteautor
\end{codigo}

\begin{codigo}[htb]
\legenda[cod:fatorial]{Função recursiva para calcular o fatorial de um número}
\begin{lstlisting}[language=Python]
def fatorial(n):
    if n == 0:
        return 1
    else:
        return n * fatorial(n - 1)
\end{lstlisting}
\fonteautor
\end{codigo}


\section{Inclusão de Algoritmos}

A formatação de algoritmos é realizada de maneira análoga à inclusão de códigos, porém, utiliza o ambiente \texttt{algoritmo}. O Algoritmo~\ref{alg:fatorial} mostra um exemplo de algoritmo para o cálculo do fatorial de maneira recursiva.

\begin{algoritmo}
\legenda[alg:fatorial]{Algoritmo para o calcular o fatorial de um número}
\begin{lstlisting}[mathescape]
ENTRADA: Número inteiro não negativo $n$
SAÍDA Valor de $n!$

SE $n = 0$
    RETORNA $1$
SENÃO
    RETURNA $n \times \text{fatorial}(n-1)$
FIM-SE
\end{lstlisting}
\fonteautor
\end{algoritmo}


\section{Considerações Finais}

O uso de algoritmos e códigos é um recurso poderoso para a formatação de trabalhos acadêmicos e científicos, especialmente em áreas técnicas. A personalização oferecida pelo LaTeX permite a adaptação dos estilos de exibição conforme as necessidades específicas do autor ou da instituição.



% -------------------------------------------------------------------------------------------------
% -------------------------------------------------------------------------------------------------
% -------------------------------------------------------------------------------------------------



% -------------------------------------------------------------------------------------------------
% 7. EQUAÇÕES E FÓRMULAS
% -------------------------------------------------------------------------------------------------
\chapter{Equações e Fórmulas}
\label{cap:equacoes}

Este capítulo apresenta como criar e formatar equações, usar comandos específicos, a numeração e referência de equações, além de explorar exemplos de equações e sistemas de equações. As equações e fórmulas matemáticas desempenham um papel essencial em diversos campos do conhecimento, especialmente nas áreas de ciências exatas, engenharias e tecnologia. Em trabalhos acadêmicos, é fundamental que a apresentação dessas expressões seja clara e bem estruturada, facilitando a comunicação precisa de conceitos matemáticos.

\section{Equações Simples}

Para a inserção de equações matemáticas, utiliza-se o ambiente \texttt{equation}. Por padrão, as equações são centralizadas e numeradas automaticamente, permitindo a referência posterior no corpo do texto. O Código~\ref{cod:eqEinstein} exemplifica a inclusão de uma equação.

\begin{codigo}[htb]
\legenda[cod:eqEinstein]{Código para inclusão de equações}
\begin{lstlisting}
\begin{equation}
    E = mc^2
    \label{eq:einstein}
\end{equation}
\end{lstlisting}
\fonteautor
\end{codigo}

A Equação~\eqref{eq:einstein} exibe a equação de Einstein. A referência a essa equação no texto pode ser realizada com o comando \texttt{\textbackslash ref}, como apresentado na frase ``conforme mostrado na Equação~\ref{eq:einstein}''.

\begin{equation}
    E = mc^2
    \label{eq:einstein}
\end{equation}

\section{Sistemas de Equações}

Para formatar sistemas de equações, utiliza-se o ambiente \texttt{\textbackslash cases}, que insere chaves ao lado do sistema, facilitando a organização de condições ou equações inter-relacionadas. O Código~\ref{cod:sistemas} exemplifica a inclusão de um equações alinhadas.

\begin{codigo}[htb]
\legenda[cod:sistemas]{Código para inclusão de equações}
\begin{lstlisting}
\begin{equation}
    f(x) =
    \begin{cases}
        x^2 & \text{se } x \geq 0 \\
        -x^2 & \text{se } x < 0
    \end{cases}
    \label{eq:sistemas}
\end{equation}
\end{lstlisting}
\fonteautor
\end{codigo}

A Equação~\eqref{eq:sistemas} mostra um exemplo de sistemas de equações. Neste exemplo, a função \( f(x) \) é definida de acordo com duas condições diferentes, dependendo do valor de \( x \). O uso do ambiente``cases'' formata o sistema de forma clara e visualmente organizada.


\begin{equation}
    f(x) =
    \begin{cases}
        x^2 & \text{se } x \geq 0 \\
        -x^2 & \text{se } x < 0
    \end{cases}
    \label{eq:sistemas}
\end{equation}


\section{Símbolos e Notações Avançadas}

O \LaTeX\ oferece uma vasta gama de símbolos e notações matemáticas, permitindo a criação de expressões complexas de forma elegante. Entre os exemplos mais comuns, incluem-se:

\begin{alineas}
    \item Frações: \verb|\frac{a}{b}| para \( \frac{a}{b} \)
    \item Somatórios: \verb|\sum_{i=1}^{n} i^2| para \( \sum_{i=1}^{n} i^2 \)
    \item Integrais: \verb|\int_0^1 x^2 \, dx| para \( \int_0^1 x^2 \, dx \)
    \item Vetores: \verb|\vec{v}| para \( \vec{v} \)
\end{alineas}

Através desses comandos, é possível construir fórmulas matemáticas de alta complexidade de maneira eficiente e esteticamente agradável.

\section{Considerações Finais}

O LaTeX é uma ferramenta robusta para a formatação de equações e fórmulas matemáticas, oferecendo recursos avançados que garantem precisão e consistência na apresentação de conceitos matemáticos. Com ambientes específicos para equações, sistemas de equações e expressões \textit{inline}, o autor possuis grande flexibilidade para a elaboração de textos acadêmicos de alta qualidade, facilitando a leitura e a compreensão dos leitores.



% -------------------------------------------------------------------------------------------------
% -------------------------------------------------------------------------------------------------
% -------------------------------------------------------------------------------------------------



% -------------------------------------------------------------------------------------------------
% 7. ALÍNEAS
% -------------------------------------------------------------------------------------------------
\chapter{Alíneas}
\label{cap:alineas}

Este capítulo apresenta o modo de uso de alíneas. Em documentos acadêmicos, o uso correto de alíneas é fundamental para garantir a clareza na apresentação de informações que não possuem um título próprio. A norma técnica da Associação Brasileira de Normas Técnicas (ABNT) orienta a estruturação adequada das alíneas de forma a facilitar a compreensão e a organização textual. O uso das alíneas segue diretrizes específicas que devem ser rigorosamente aplicadas, conforme descrito a seguir~\cite{abnt2012}.

As alíneas são empregadas quando é necessário enumerar vários assuntos em uma seção sem título. Um exemplo de alíneas pode ser observado na classificação dos paradigmas de programação. Os principais paradigmas de programação são:

\begin{alineas}
    \item Programação imperativa;
    \item Programação funcional;
    \item Programação orientada a objetos;
    \item Programação lógica:
        \begin{subalineas}
            \item baseada no uso de lógica formal para expressar programas;
            \item foca em resolver problemas declarando relações e permitindo que o computador derive soluções;
            \item exemplos comuns incluem linguagens como Prolog.
        \end{subalineas}
\end{alineas}

Neste exemplo, observa-se o uso correto de alíneas para organizar a informação de maneira hierárquica e clara. O uso de subalíneas foi aplicado para detalhar a ``Programação lógica''. Este tipo de estruturação facilita a compreensão e o destaque de elementos dentro de uma lista de assuntos.



% -------------------------------------------------------------------------------------------------
% -------------------------------------------------------------------------------------------------
% -------------------------------------------------------------------------------------------------



% -------------------------------------------------------------------------------------------------
% 8. CRONOGRAMA
% -------------------------------------------------------------------------------------------------
\chapter{Cronograma}
\label{cap:cronograma}

O cronograma é uma ferramenta essencial no planejamento de um Trabalho de Conclusão de Curso~(TCC). Ele representa a programação das atividades necessárias para a conclusão do projeto, facilitando a organização e o gerenciamento do tempo. Um cronograma bem elaborado é fundamental para garantir que todas as etapas do TCC sejam realizadas dentro do prazo, evitando sobrecargas de trabalho.

O cronograma deve incluir as principais etapas do TCC, como a definição do tema, revisão bibliográfica, desenvolvimento da metodologia, coleta e análise de dados, redação do trabalho e entrega final. O Quadro~\ref{tab:cronograma} apresenta um exemplo de cronograma.

\begin{quadro}[h]
    \legenda[tab:cronograma]{Cronograma detalhado das atividades do TCC}
    \begin{tabular}{|p{7cm}|c|c|c|c|c|}
        \hline
        \textbf{Atividade}                      & \textbf{Ago} & \textbf{Set} & \textbf{Out} & \textbf{Nov} & \textbf{Dez} \\
        \hline\hline
        Definição do Tema                      & \cellcolor{black}  &  &  &  &  \\\hline
        Revisão Bibliográfica                  & \cellcolor{black}  & \cellcolor{black}  &  &  &  \\\hline
        Desenvolvimento da Metodologia         &  & \cellcolor{black}  & \cellcolor{black}  &  &  \\\hline
        Coleta de Dados                        &  &  & \cellcolor{black}  & \cellcolor{black}  &  \\\hline
        Análise dos Dados                      &  &  &  & \cellcolor{black}  & \cellcolor{black}  \\\hline
        Redação do TCC                        &  &  &  & \cellcolor{black}  & \cellcolor{black}  \\\hline
        Revisão Final e Entrega                &  &  &  &  & \cellcolor{black}  \\
        \hline
    \end{tabular}
    \vspace{0.2cm}
    \fonteautor
\end{quadro}




% -------------------------------------------------------------------------------------------------
% 9. CONCLUSÃO
% -------------------------------------------------------------------------------------------------
\chapter{Conclusão}
\label{cap:conclusao}

A conclusão é uma das partes mais relevantes de um Trabalho de Conclusão de Curso~(TCC), pois é nela que o autor sintetiza os principais achados da pesquisa e apresenta suas reflexões finais. O objetivo principal da conclusão é proporcionar uma visão clara e objetiva sobre os resultados obtidos, além de contextualizá-los dentro da problemática abordada no trabalho. Para isso, uma conclusão eficaz deve seguir algumas diretrizes, garantindo que todos os pontos cruciais sejam abordados.

Primeiramente, a conclusão deve retomar os objetivos do trabalho. Ao revisitar as perguntas de pesquisa e os objetivos estabelecidos na introdução, o autor consegue avaliar se foram alcançados ou não. Este processo de revisão não apenas reforça a relevância do tema, mas também permite que o leitor entenda como as informações apresentadas ao longo do texto se conectam com o que foi inicialmente proposto. Segundo \citeonline{gil}, a conclusão deve ``revisitar os objetivos traçados, avaliando o quanto se conseguiu avançar na pesquisa''~\cite{gil}.

Em segundo lugar, é fundamental que a conclusão contenha uma síntese dos principais resultados e achados da pesquisa. Essa síntese deve ser objetiva e direta, evitando a inclusão de novos dados ou informações não discutidas previamente no corpo do texto. A análise crítica dos resultados obtidos pode ser feita de maneira a apontar as implicações e contribuições que o trabalho traz para o campo de estudo em questão. Como afirma~\citeauthoronline{lakatos}, ``a conclusão é o momento de apresentar as implicações do estudo realizado, bem como suas contribuições para a área de conhecimento''~\cite{lakatos}.

Por fim, a conclusão pode incluir sugestões para futuras pesquisas. Este aspecto é importante porque pode abrir novas possibilidades de investigação, além de indicar áreas que precisam de mais atenção ou aprofundamento. A inclusão de sugestões é uma forma de reconhecer que o conhecimento é um processo contínuo e que existem lacunas a serem preenchidas. Como destaca \citeonline{lakatos}, ``sugerir novos caminhos de pesquisa é uma maneira de contribuir para o desenvolvimento da área e para a construção do conhecimento''~\cite{lakatos}.

Em suma, a conclusão deve retomar os objetivos, sintetizar os resultados e sugerir novas pesquisas. Essa estrutura não apenas ajuda o leitor a entender o trabalho de forma mais clara, mas também enfatiza a contribuição do autor para a área estudada.



% -------------------------------------------------------------------------------------------------
% -------------------------------------------------------------------------------------------------
% -------------------------------------------------------------------------------------------------



% =================================================================================================
% -------------------------------------------------------------------------------------------------
% - ELEMENTOS PÓS-TEXTUAIS                                                                            -
% -------------------------------------------------------------------------------------------------
% =================================================================================================

% Inicia a seção pós-textual
\postextual

% -------------------------------------------------------------------------------------------------
% REFERÊNCIAS BIBLIOGRÁFICAS
% -------------------------------------------------------------------------------------------------

% Carrega o arquivo referencias.bib
\bibliography{referencias}


% -------------------------------------------------------------------------------------------------
% APÊNDICES
% -------------------------------------------------------------------------------------------------
\apendice[ape:exemplo]{Exemplo de Apêndice}

Os apêndices e anexos são elementos complementares em um Trabalho de Conclusão de Curso~(TCC) que têm como objetivo fornecer informações adicionais relevantes ao entendimento da pesquisa. Embora sejam frequentemente utilizados de forma intercambiável, eles possuem finalidades distintas e devem ser utilizados em contextos específicos. A correta utilização de apêndices e anexos contribui para a clareza e a organização do trabalho acadêmico.

Os apêndices são utilizados para incluir informações que são de autoria do próprio autor do TCC. Isso pode incluir questionários, roteiros de entrevistas, tabelas com dados adicionais e qualquer outro material que seja essencial para a compreensão do trabalho, mas que não é possível inserir diretamente no corpo do texto. Segundo~\citeonline{severino}, ``os apêndices servem para apresentar materiais elaborados pelo autor que complementam a sua pesquisa''~\cite{severino}. Dessa forma, o uso de apêndices é fundamental quando se deseja apresentar materiais que, embora relevantes, não se encaixam na narrativa principal do trabalho.


% -------------------------------------------------------------------------------------------------
% ANEXOS
% -------------------------------------------------------------------------------------------------
\anexo[ane:exemplo]{Exemplo de Anexo}

Os anexos são documentos que não são de autoria do autor do TCC, mas que são pertinentes à pesquisa. Eles podem incluir gráficos, documentos oficiais, normativas e textos que foram utilizados como referência ou que fornecem suporte à pesquisa. Segundo a Associação Brasileira de Normas Técnicas (ABNT), ``os anexos são documentos que, embora não sejam parte integrante do texto, são relevantes para a compreensão do tema tratado''~\cite{abnt2012}. O uso de anexos é indicado quando se deseja fornecer ao leitor informações complementares que não foram geradas pelo autor, mas que ajudam a esclarecer ou contextualizar a pesquisa.

Ambos, apêndices e anexos, devem ser listados em uma página específica após a conclusão do trabalho, de forma a facilitar a consulta por parte do leitor. A correta distinção entre apêndices e anexos, bem como o seu uso apropriado, é essencial para a organização e a credibilidade do TCC. Ao fornecer informações adicionais de maneira clara e sistemática, o autor contribui para uma melhor compreensão do seu trabalho, enriquecendo a experiência do leitor.


\end{document}
